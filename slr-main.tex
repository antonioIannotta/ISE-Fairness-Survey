\documentclass{article}
\usepackage{geometry}
\usepackage{graphicx}
\usepackage{algorithm2e}
\usepackage{hyperref}
\usepackage{amsmath}
\usepackage{multirow}
\usepackage{caption}
\usepackage{float}

\geometry{a4paper, left=25mm, right=25mm, top=30mm, bottom=30mm}

\title{Systematic Scientific Review on Fairness}
\author{Antonio Iannotta}
\date{\today}

\begin{document}

\maketitle

\begin{abstract}
    The widespread integration of intelligent systems based on artificial intelligence (AI) and machine learning (ML) in sectors such as healthcare, finance, and justice has raised concerns about fairness and discrimination. This Systematic Literature Review (SLR) aims to examine existing research on "fairness" in intelligent systems, with a focus on challenges, proposed solutions, and tools used to address bias in AI/ML-based systems.
    The SLR reviews a wide range of scientific publications, including articles, empirical studies, and guidelines, to provide a comprehensive overview of the current state of research. Commonly used approaches and metrics for measuring and mitigating bias in data and algorithms are examined. Additionally, ethical and social challenges associated with the implementation of fair intelligent systems are analyzed, along with key recommendations.
    This work contributes to the understanding of fairness challenges in intelligent systems and serves as a reference for future research and development in this crucial field to ensure fairer and more inclusive systems. \\
    \textbf{Keywords:} fairness, intelligent systems, artificial intelligence, machine learning, discrimination, bias.
\end{abstract}

\section{Introduction}
The widespread integration of artificial intelligence (AI) systems across critical domains, including healthcare, finance, and justice, has introduced new opportunities while raising critical questions about fairness and discrimination. In the real world, AI systems may inherit biases and discrimination from the training data, jeopardizing fairness and objectivity in decision-making. The quest for solutions to mitigate bias in AI systems and ensure fairness has become a paramount concern. \\
This Systematic Literature Review (SLR) aims to comprehensively examine the evolution of approaches designed to ensure fairness in AI systems over the past decade. The issue of fairness has emerged as a central theme in the discourse on AI responsibility and ethics, as it directly influences the societal impact of these systems. \\
Fairness is a fundamental concept in ensuring that AI systems are just and equitable. Discrimination and bias can lead to unjust decisions with significant consequences for those affected. Addressing these issues is imperative to ensure that technological advancements in AI are inclusive and uphold fundamental rights. \\
In this SLR, we seek to address the following research question: How have approaches to ensuring fairness in AI systems evolved over the years? We will examine research progress, newly developed methodologies, and outstanding challenges in this rapidly evolving field. \\
This study focuses on the past decade, aiming to capture the most recent developments and emerging trends concerning fairness in AI systems. \\
This SLR assumes critical importance as it provides an up-to-date and in-depth analysis of the methodologies and approaches used to ensure fairness in AI systems. Such an analysis is pivotal for understanding how AI can be developed responsibly and in alignment with ethical principles.

\section{Methodology}
    \subsection{Search Strategy}
    \textbf{Database Searches:}
\begin{itemize}
    \item ACM Digital Library
    \item Google Scholar
\end{itemize}

\textbf{Search Strings:}
\begin{enumerate}
    \item "Approaches for mitigating bias in artificial intelligence systems"
    \item "Methods for ensuring fairness in AI systems"
    \item "Techniques to reduce bias in machine learning"
    \item "AI system bias mitigation methodologies"
    \item "Fairness and bias in artificial intelligence"
    \item "Ethical AI and bias reduction"
    \item "AI system fairness over the last 10 years"
    \item "Bias mitigation strategies in AI"
\end{enumerate}

\textbf{Inclusion and Exclusion Criteria:}
\begin{itemize}
    \item Publication Period: Last 10 years (from 2013 to 2023).
    \item Language: English.
    \item Document Types: Scientific articles, conference papers, conference proceedings, and relevant academic documents.
    \item Excluded: Duplicate documents and articles not relevant to the research topic.
\end{itemize}

This search strategy is designed to retrieve relevant articles published in the last 10 years that address approaches for mitigating bias and ensuring fairness in artificial intelligence systems. Make sure to specify the start and end dates for the publication period. You can use these search strings in the specified databases to identify pertinent sources for your Systematic Literature Review.

    \subsection{Study Selection}
        % Study selection process details go here...

    \subsection{Data Extraction}
        % Data extraction process details go here...

    \subsection{Quality Assessment}
        % Quality assessment criteria and process details go here...

    \subsection{Data Synthesis}
        % Data synthesis methods and approach details go here...

    \subsection{Reporting Guidelines}
        % Mention if any specific reporting guidelines were followed...

\section{Results}
    % Results summary, tables, charts, and key findings go here...

\section{Discussion}
    % Interpretation of results, implications, limitations, and future research areas go here...

\section{Conclusion}
    % Summarize the main findings and conclusions go here...

\section{References}
    % List of cited studies and sources go here...

\section{Appendices}
    % Include supplementary materials if necessary...

\section{Acknowledgments}
    % Acknowledgments go here...

\end{document}